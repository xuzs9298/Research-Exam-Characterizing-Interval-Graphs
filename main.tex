\documentclass[a4paper,12pt]{article}

%% Language and font encodings
\usepackage[english]{babel}
\usepackage[utf8x]{inputenc}
\usepackage[T1]{fontenc}
\usepackage[ruled,linesnumbered]{algorithm2e}
\usepackage{float}
\usepackage{import}

%% Sets page size and margins
\usepackage[a4paper,top=3cm,bottom=2cm,left=3cm,right=3cm,marginparwidth=1.75cm]{geometry}

%% Useful packages
\usepackage{amsmath}
\usepackage{graphicx}
\usepackage[colorinlistoftodos]{todonotes}
\usepackage[colorlinks=true, allcolors=blue]{hyperref}
\usepackage{amsmath,amsfonts,amssymb,amsthm,epsfig,epstopdf,titling,url,array}
\usepackage{natbib}

\theoremstyle{plain}
\newtheorem{case}{Case}
\newtheorem{theorem}{Theorem}[section]
\newtheorem{lemma}[theorem]{Lemma}
\newtheorem{proposition}[theorem]{Proposition}
\newtheorem{corollary}[theorem]{Corollary}

\makeatletter
\newcommand*{\rom}[1]{\expandafter\@slowromancap\romannumeral #1@}
\makeatother

\theoremstyle{definition}
\newtheorem{definition}{Definition}[section]
\newtheorem{conjecture}{Conjecture}[section]
\newtheorem{example}{Example}[section]


\title{Characterizing Interval Graphs}
\author{ZHISHENG XU}
\date{}

\begin{document}
\maketitle


\section{Abstract}
An undirected graph is an $interval\;graph$ if we can create a model with an interval for each vertex
and two vertices $v_i$ and $v_j$ are adjacent whenever the two corresponding intervals intersect.
As an important subclass of perfect graphs with rich structures and history, the class of interval graphs has been studied for decades
and has applications in various combinatorial optimization and decision problems. It is also closely related to other graph classes such as chordal graphs, comparability graphs, and permutation graphs.\paragraph{}
This exam will explore its properties in the following aspects:
\begin{enumerate}
  \item Chordal graphs and interval graph characterizations.
  \item A linear time recognition algorithm.
  \item Forbidden structures and how to find them.
\end{enumerate}

Initial Papers: \cite{rose1976algorithmic}, \cite{lekkerkerker1962representation}, \cite{lindzey2016linear}, \cite{tucker1972structure}, \cite{booth1976testing}
\newpage
\tableofcontents
\newpage





\import{sections/}{introduction.tex}
\import{sections/}{notations.tex}


\import{sections/}{chordal.tex}
\import{sections/}{interval_recognition.tex}
\import{sections/}{forbidden_structure.tex}





\bibliographystyle{unsrt}
\bibliography{reference}

%\input{reference.bbl}
\end{document}